%% Le lingue utilizzate, che verranno passate come opzioni al pacchetto babel. Come sempre, l'ultima indicata sarà quella primaria.
%% Se si utilizzano una o più lingue diverse da "italian" o "english", leggere le istruzioni in fondo.
\def\thudbabelopt{english,italian}
%% Valori ammessi per target: bach (tesi triennale), mst (tesi magistrale), phd (tesi di dottorato).
%% Valori ammessi per aauheader: '' (vuoto -> nessun header Alpen Adria Univeristat), aics (Department of Artificial Intelligence and Cybersecurity), informatics (Department of Informatics Systems). Il nome del dipartimento è allineato con la versione inglese del logo UniUD.
%% Valori ammessi per style: '' (vuoto -> stile moderno), old (stile tradizionale).
\documentclass[target=bach,aauheader=,style=]{thud}

%% --- Informazioni sulla tesi ---
%% Per tutti i tipi di tesi
% Scommentare quello di interesse, o mettete quello che vi pare
\course{Informatica}
%\course{Internet of Things, Big Data e Web}
%\course{Matematica}
%\course{Comunicazione Multimediale e Tecnologie dell'Informazione}
\title{Un sistema distribuito per l'analisi della topologia di reti di calcolatori}
\author{Diego Cirillo}
\supervisor{Prof.\ Marino Miculan}
\cosupervisor{Dott.\ Matteo Paier}
%\tutor{Guido Necchi}
%% Campi obbligatori: \title, \author e \course.
%% Altri campi disponibili: \reviewer, \tutor, \chair, \date (anno accademico, calcolato in automatico), \rights
%% Con \supervisor, \cosupervisor, \reviewer e \tutor si possono indicare più nomi separati da \and.
%% Per le sole tesi di dottorato:
%\phdnumber{313}
%\cycle{XXVIII}
%\contacts{Via della Sintassi Astratta, 0/1\\65536 Gigatera --- Italia\\+39 0123 456789\\\texttt{http://www.example.com}\\\texttt{inbox@example.com}}

%% --- Pacchetti consigliati ---
%% pdfx: per generare il PDF/A per l'archiviazione. Necessario solo per la versione finale
\usepackage[a-1b]{pdfx}
%% hyperref: Regola le impostazioni della creazione del PDF... più tante altre cose. Ricordarsi di usare l'opzione pdfa.
\usepackage[pdfa]{hyperref}
%% tocbibind: Inserisce nell'indice anche la lista delle figure, la bibliografia, ecc.
\graphicspath{ {./images/} }
%% --- Stili di pagina disponibili (comando \pagestyle) ---
%% sfbig (predefinito): Apertura delle parti e dei capitoli col numero grande; titoli delle parti e dei capitoli e intestazioni di pagina in sans serif.
%% big: Come "sfbig", solo serif.
%% plain: Apertura delle parti e dei capitoli tradizionali di LaTeX; intestazioni di pagina come "big".
\usepackage{placeins}

\begin{document}
\maketitle

%% Dedica (opzionale)
%\begin{dedication}
%	Al mio cane,\par per avermi ascoltato mentre ripassavo le lezioni.
%\end{dedication}

%% Ringraziamenti (opzionali)
%\acknowledgements
%Sed vel lorem a arcu faucibus aliquet eu semper tortor. Aliquam dolor lacus, semper vitae ligula sed, blandit iaculis leo. Nam pharetra lobortis leo nec auctor. Pellentesque habitant morbi tristique senectus et netus et malesuada fames ac turpis egestas. Fusce ac risus pulvinar, congue eros non, interdum metus. Mauris tincidunt neque et aliquam imperdiet. Aenean ac tellus id nibh pellentesque pulvinar ut eu lacus. Proin tempor facilisis tortor, et hendrerit purus commodo laoreet. Quisque sed augue id ligula consectetur adipiscing. Vestibulum libero metus, lacinia ac vestibulum eu, varius non arcu. Nam et gravida velit.

%% Sommario (opzionale)
\abstract
La crescente complessità delle reti di calcolatori moderne richiede strumenti sofisticati per la loro analisi e gestione. Questa tesi presenta la progettazione e lo sviluppo di un sistema distribuito innovativo, capace di esplorare e mappare in modo efficiente le topologie di reti di grandi dimensioni. Il sistema proposto è in grado di estrarre informazioni significative sulla struttura della rete, identificando pattern, anomalie e potenziali vulnerabilità. L'obiettivo finale è fornire ai network administratori un quadro completo e dettagliato della loro infrastruttura, supportando decisioni informate e ottimizzando le performance della rete.

%% Indice
\tableofcontents

%% Lista delle tabelle (se presenti)
%\listoftables

%% Lista delle figure (se presenti)
\listoffigures
!TODO

%% Corpo principale del documento
\mainmatter

%% Parte
%% La suddivisione in parti è opzionale; solitamente sono sufficienti i capitoli.
%\part{Parte}

%% Capitolo
\chapter{Introduzione}
Le reti informatiche moderne, caratterizzate da una crescente complessità e eterogeneità, richiedono strumenti di gestione sempre più sofisticati. Per decenni, il Simple Network Management Protocol (SNMP) ha rappresentato lo standard de facto per il monitoraggio e la gestione delle reti. Tuttavia, l'SNMP, pur essendo un protocollo ampiamente diffuso, presenta una serie di limitazioni che ne compromettono l'efficacia in ambienti altamente dinamici e distribuiti.

La natura centralizzata dell'SNMP, basata su un modello client-server, lo rende vulnerabile a punti di singolo fallimento e difficoltà di scalabilità in reti di grandi dimensioni. Inoltre, l'SNMP è stato progettato per un'era in cui i dispositivi di rete erano relativamente semplici e omogenei. Oggi, la proliferazione di dispositivi IoT, cloud computing e virtualizzazione ha reso le reti molto più complesse, richiedendo un approccio più flessibile e adattivo.

Questa tesi presenta la progettazione e l'implementazione di un sistema distribuito innovativo, appositamente concepito per affrontare le complessità dell'analisi della topologia di rete. Sfruttando le potenzialità del calcolo distribuito e di algoritmi avanzati, il sistema proposto è in grado di esplorare e mappare in modo efficiente reti di qualsiasi dimensione e complessità, fornendo una rappresentazione visuale e analitica dettagliata della loro struttura.

L'obiettivo principale di questa ricerca è sviluppare uno strumento che consenta ai network administratori di:

\begin{itemize}
  \item \textbf{Ottenere una visione olistica della rete}: Visualizzare la topologia della rete in modo intuitivo, identificando facilmente i punti di connessione, i dispositivi critici e le eventuali anomalie.
  \item \textbf{Identificare i colli di bottiglia}: Individuare le parti della rete soggette a congestione, ottimizzando così l'allocazione delle risorse.
  \item \textbf{Prevenire i guasti}: Monitorare lo stato dei dispositivi e delle connessioni, rilevando potenziali problemi prima che si verifichino.
  \item \textbf{Facilitare la risoluzione dei problemi}: Utilizzare le informazioni sulla topologia per isolare rapidamente le cause di malfunzionamenti e pianificare interventi di manutenzione.
\end{itemize}

Il sistema proposto si distingue per le seguenti caratteristiche:

\begin{itemize}
  \item \textbf{Scalabilità}: Capacità di gestire reti di grandi dimensioni e in continua evoluzione.

  \item \textbf{Efficienza}: Riduzione dei tempi di analisi grazie all'utilizzo di algoritmi paralleli e distribuiti.
    
  \item \textbf{Flessibilità}: Adattabilità a diverse tipologie di reti e protocolli di comunicazione.

  \item \textbf{Intuitività}: Interfaccia utente user-friendly per una facile interpretazione dei risultati.
\end{itemize}

%%% Sezione
%\section{Titolo della Sezione}
%Donec pulvinar neque non lectus vulputate pellentesque. Quisque rutrum arcu velit, in feugiat sapien posuere vel. Praesent metus orci, aliquam ac cursus eget, fermentum a nisl. Etiam eu augue lacus. Nam nisi sapien, mattis sed vehicula non, pellentesque at quam. Sed euismod, dolor nec commodo lobortis, erat erat ultricies eros, bibendum dictum nulla felis in dui. Nulla blandit ultrices arcu, vitae lacinia tellus tempor sit amet. Nulla non tincidunt dolor. In eget luctus sem, sed elementum ligula. Proin elementum adipiscing sem, sit amet ultricies nisl tincidunt eu. Ut lobortis dui quam, et scelerisque erat ultrices sit amet. Sed libero sem, mollis quis euismod quis, suscipit ac justo.
%
%%% Sottosezione
%\subsection{Sottosezione}
%Donec cursus tortor eget sem ornare imperdiet. Ut vel orci non ipsum condimentum laoreet vitae ut sapien. Aenean metus mi, vehicula quis turpis nec, porttitor blandit dui. Nullam sed sollicitudin quam. Fusce nisl ante, commodo eget lacus ac, mollis ullamcorper neque. Quisque faucibus dictum nisl, dignissim fermentum sapien fringilla vel. Proin dui velit, molestie sit amet sapien et, pellentesque tristique purus. Curabitur ac quam ac diam varius bibendum.
%

%% Capitolo
\chapter{Stato dell'arte}
\label{art}
\section{Premessa}
Questo capitolo è dedicato alla specificazione della architettura generale del sistema prima delle modifiche 

\section{Architettura}
Il sistema è composto da una architettura client-server nella quale il backend si occupa di fare le scansioni e di salvarne i risultati e di fornirli al frontend dove possono essere visualizzati dagli utenti dovrebbero interagire solo con esso. 


\begin{figure}[h]
  \includegraphics[width=15cm, height=10cm]{client_server}
  \centering
\end{figure}

\FloatBarrier

Per organizzazione e scalabilità abbiamo diviso l'architettura in 6 moduli principali.


\begin{figure}[h]
  \includegraphics[width=15cm, height=10cm]{moduli_new}
  \centering
\end{figure}

\FloatBarrier

\subsection{Scan Probes}
\textbf{Scan Probes} sono moduli che vengono collocati in giro per la rete a raccogliere informazoini. 
Ricevuto un comando il probe lo interpreta, esegue l'opportuna scansione e deserializza il 'output dello scan.


\begin{figure}[h]
  \includegraphics[width=14cm, height=8cm]{probe}
  \centering
\end{figure}
\FloatBarrier


\subsection{Dispatcher} 
Il \textbf{Dispatcher} è il modulo che controlla e coordina le operazioni necessarie per le esecuzioni delle scansioni.
Riceve un comando dal Frontend che viene interpretato e mandato allo Scheduler che interagisce con l'Executer ed il Merger per eseguire la scansione e salvarla.

\begin{figure}[h]
  \includegraphics[width=14cm, height=8cm]{dispatcher}
  \centering
\end{figure}

\FloatBarrier

\subsection{Executer} 
L'\textbf{Executer} riceve un comando dallo scheduler e coordina gli Scan Probes nell'esecuzione della scansione per poi accumulare i loro risultati e mandarli al Knowledge Base Connector.

\begin{figure}[h]
  \includegraphics[width=14cm, height=8cm]{executer}
  \centering
\end{figure}

\FloatBarrier

\newpage

\subsection{Merger}
Il \textbf{Merger} si occupa di ricevere i dati generati dai Scan Probes ed amalgamarli per creare un'immagine comprensibile della rete analizzata. 


\begin{figure}[h]
  \includegraphics[width=14cm, height=8cm]{merger}
  \centering
\end{figure}

\FloatBarrier

\chapter{Implementazione}

\section{Identificazione univoca dei probe}
Per distinguire i probe abbiamo deciso di utilizzare il loro indirizzo MAC (Medium Access Controll)ovvero un indirizzo di 12 caratteri alfanumerici (es. 00:11:22:33:44:55) che viene assegnato ad ogni dispositivo di rete e indentifica in modo univoco la scheda di rete del dispositivo.

\section{Parsing XML}
Lo scan di NMAP ritorna uno stream di dati in XML molto grande ed aspettare che l'intero stream di dati finisca per poi effettuare il parsing può essere pericoloso (es. va via la corrente e si perdono ore di computazione). Analizzando la struttura del XML che viene generato da NMAP tramite l'apposito DTD (Document Type Definition) abbiamo dedotto che fare un parsing incrementale su ogni tag host è la soluzione più ottimale per diminuire possibili problemi.

\section{Configurazione}
!TODO

\section{Software usati}

\subsection{Backbone di comunicazione}
Per comunicare con i probe situati per la rete viene usato MQTT (Message Queuing Telemtry Transport). 
MQTT è un protocollo di messaggistica leggero e public/subscribe progettato per l'Internet of Things (IoT).
Come broker MQTT viene usato Mosquitto per il suo supporto al Qualify of Service livello 2 che garantisce la recezione di messaggi exactly-once fra i probes e il sistema.

\subsection{Scanner}
I probe per eseguire le analisi usano NMAP che è una utility gratis ed open source per fare network discovery anche usata da moltissimi amministratori di reti per fare inventari di rete e monitorare host e servizi. 

\subsection{Database}
!TODO

\subsection{Tauri}
Tauri è un framework innovativo che permette di creare applicazioni desktop utilizzando tecnologie web familiari come HTML, CSS, JS. L'applicazione web, invece di venire eseguita in un browser, è trasformata in un'app desktop completa, con tutte le funzionalità native di un'applicazione tradizionale.
Per il frontend abbiamo scelto di usare principalmente React, e per il backend Rust.

\subsection{Rust}
Rust è un linguaggio di programmazione moderno, compliato e multi-paradigma, sviluppato da Mozilla Research in collaborazoine con la comunità open-source. Nato con l'obiettivo di coniugare efficienza, sicurezza e affidabilità, Rust si distingue per le sue caratteristiche peculiari:
\begin{itemize}
  \item \textbf{Sicurezza della memoria}: Rust elimina la possibilità di dangling pointers e memory leaks, tipici problemi di altri linguaggi che causano crash e comportamenti inaspettati.
  \item \textbf{Prestazioni elevate}: Rust offre prestazioni paragonabili a quelle del C++, grazie alla compilazione in codice macchina efficiente e all'assenza di runtime overhead.
  \item \textbf{Concorrenza sicura}: Rust gestisce la concorrenza in modo nativo, prevenendo automaticamente i data race e garantendo la sicurezza dei thread senza la necessità di complessi meccanismi di sincronizzazione.
  \item \textbf{Produttività}: Rust offre un sistema di ownership moderno e un potente sistema di tipi che aiutano a scrivere codice pulito, conciso e robusto.
\end{itemize}


\subsection{TypeScript}
TypeScript è un superset di JavaScript, ovvero un linguaggio che estende le sue funzionalità con tipi statici, classi, interfacce e moduli opzionali. È stato sviluppato da Microsoft e rilasciato nel 2012, guadagnando rapidamente popolarità nella community di sviluppo web.
Dei punti chiave di TypeScript sono:


\section{Frameworks e librerie}
\subsection{Rust}
Le librerie principali usate in Rust sono:
\begin{itemize}
  \item \textbf{Serde}: un framwork per la serializzazione e la deserializzazione per Rust, che consente di convertire facilmente dati strutturati in formati binari come JSON, BSON, XML ed altri. 
  \item \textbf{Tokio}: un runtime asincrono per Rust. Fornisce i mattoni essenziali per la scrittura di applicazioni di rete offrendo elevata velocità, scalabilità, facilità d'uso, affidabilità, robustezza e flessibilità.
  \item \textbf{Tracing} è uno strumento prezioso per la risoluzione dei problemi, il debug e il monitoraggio delle prestazioni di applicazioni Rust complesse.
\end{itemize}

\subsection{TypeScript}
La libreria principale per l'interfaccia grafica è React, una libreria open-source sviluppata da Meta per la creazoine di interfacce utente dinamiche e perfomanti. 
Permette di costruire UI complesse scomponendole in componenti riutilizzabili, facilidando lo sviluppo e la manutenzoine di applicazioni web.

\begin{itemize}
  \item \textbf{Tipizzazione statica}: TypeScript consente di definire i tipi di dati per variabili e funzioni, migliorando la leggibilità, la manutenibilità e la robustezza del codice. Il compilatore di TypeScript verifica che i tipi vengano utilizzati correttamente, aiutando a prevenire errori durante l'esecuzione.
  \item \textbf{Maggiori funzionalità}: TypeScript introduce concetti come classi, interfacce e moduli, che facilitano la scrittura di codice strutturato, modulare e scalabile. Queste funzionalità sono particolarmente utili per progetti di grandi dimensioni.
  \item \textbf{Compilazione in JavaScript}: Il codice TypeScript viene compilato in JavaScript puro, che può essere eseguito da qualsiasi browser o motore JavaScript. Questo significa che è possibile utilizzare TypeScript per sviluppare applicazioni web moderne senza preoccuparsi della compatibilità.
\end{itemize}


%% Capitolo
\chapter{Lavoro principale}
In hac habitasse platea dictumst. Vestibulum consectetur dictum pellentesque. Suspendisse nunc neque, commodo ac imperdiet nec, sollicitudin vitae libero. Donec bibendum vel nunc vitae pharetra. In vel volutpat odio, et interdum dui. Duis mauris ligula, congue eget molestie at, tincidunt nec diam. Nam vitae eros nec arcu suscipit vehicula. Aliquam consectetur imperdiet elit, eget pretium arcu fringilla at. Maecenas \cite{Knu86} sed libero pulvinar, mattis tortor vel, fermentum enim.


%% Capitolo
\chapter{Risultati}
In hac habitasse platea dictumst. Vestibulum consectetur dictum pellentesque. Suspendisse nunc neque, commodo ac imperdiet nec, sollicitudin vitae libero. Donec bibendum vel nunc vitae pharetra. In vel volutpat odio, et interdum dui. Duis mauris ligula, congue eget molestie at, tincidunt nec diam. Nam vitae eros nec arcu suscipit vehicula. Aliquam consectetur imperdiet elit, eget pretium arcu fringilla at. Maecenas \cite{Knu86} sed libero pulvinar, mattis tortor vel, fermentum enim.


%% Capitolo
\chapter{Conclusioni}
In hac habitasse platea dictumst. Vestibulum consectetur dictum pellentesque. Suspendisse nunc neque, commodo ac imperdiet nec, sollicitudin vitae libero. Donec bibendum vel nunc vitae pharetra. In vel volutpat odio, et interdum dui. Duis mauris ligula, congue eget molestie at, tincidunt nec diam. Nam vitae eros nec arcu suscipit vehicula. Aliquam consectetur imperdiet elit, eget pretium arcu fringilla at. Maecenas \cite{Knu86} sed libero pulvinar, mattis tortor vel, fermentum enim.

%% Fine dei capitoli normali, inizio dei capitoli-appendice (opzionali)
\appendix

%\part{Appendici}

\chapter{Titolo della prima appendice}
Sed purus libero, vestibulum ut nibh vitae, mollis ultricies augue. Pellentesque velit libero, tempor sed pulvinar non, fermentum eu leo. Duis posuere eleifend nulla eget sagittis. Nam laoreet accumsan rutrum. Interdum et malesuada fames ac ante ipsum primis in faucibus. Curabitur eget libero quis leo porttitor vehicula eget nec odio. Proin euismod interdum ligula non ultricies. Maecenas sit amet accumsan sapien.

%% Parte conclusiva del documento; tipicamente per riassunto, bibliografia e/o indice analitico.
\backmatter

%% Riassunto (opzionale)
%\summary
%Maecenas tempor elit sed arcu commodo, dapibus sagittis leo egestas. Praesent at ultrices urna. Integer et nibh in augue mollis facilisis sit amet eget magna. Fusce at porttitor sapien. Phasellus imperdiet, felis et molestie vulputate, mauris sapien tincidunt justo, in lacinia velit nisi nec ipsum. Duis elementum pharetra lorem, ut pellentesque nulla congue et. Sed eu venenatis tellus, pharetra cursus felis. Sed et luctus nunc. Aenean commodo, neque a aliquam bibendum, mauris augue fringilla justo, et scelerisque odio mi sit amet diam. Nulla at placerat nibh, nec rutrum urna. Donec ut egestas magna. Aliquam erat volutpat. Phasellus vestibulum justo sed purus mattis, vitae lacinia magna viverra. Nulla rutrum diam dui, vel semper mi mattis ac. Vestibulum ante ipsum primis in faucibus orci luctus et ultrices posuere cubilia Curae; Donec id vestibulum lectus, eget tristique est.

%% Bibliografia (praticamente obbligatoria)
\bibliographystyle{plain_\languagename}%% Carica l'omonimo file .bst, dove \languagename è la lingua attiva.
%% Nel caso in cui si usi un file .bib (consigliato)
\bibliography{thud}
%% Nel caso di bibliografia manuale, usare l'environment thebibliography.

%% Per l'indice analitico, usare il pacchetto makeidx (o analogo).

\end{document}

--- Istruzioni per l'aggiunta di nuove lingue ---
Per ogni nuova lingua utilizzata aggiungere nel preambolo il seguente spezzone:
    \addto\captionsitalian{%
        \def\abstractname{Sommario}%
        \def\acknowledgementsname{Ringraziamenti}%
        \def\authorcontactsname{Contatti dell'autore}%
        \def\candidatename{Candidato}%
        \def\chairname{Direttore}%
        \def\conclusionsname{Conclusioni}%
        \def\cosupervisorname{Co-relatore}%
        \def\cosupervisorsname{Co-relatori}%
        \def\cyclename{Ciclo}%
        \def\datename{Anno accademico}%
        \def\indexname{Indice analitico}%
        \def\institutecontactsname{Contatti dell'Istituto}%
        \def\introductionname{Introduzione}%
        \def\prefacename{Prefazione}%
        \def\reviewername{Controrelatore}%
        \def\reviewersname{Controrelatori}%
        %% Anno accademico
        \def\shortdatename{A.A.}%
        \def\summaryname{Riassunto}%
        \def\supervisorname{Relatore}%
        \def\supervisorsname{Relatori}%
        \def\thesisname{Tesi di \expandafter\ifcase\csname thud@target\endcsname Laurea\or Laurea Magistrale\or Dottorato\fi}%
        \def\tutorname{Tutor aziendale%
        \def\tutorsname{Tutor aziendali}%
    }
sostituendo a "italian" (nella 1a riga) il nome della lingua e traducendo le varie voci.
